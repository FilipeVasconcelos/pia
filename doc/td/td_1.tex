\documentclass[11pt,a4paper]{article} %twocolumn
\usepackage[francais]{babel}
\usepackage[utf8]{inputenc}
\usepackage[T1]{fontenc}
\usepackage{lmodern}


\usepackage{amsmath}
\usepackage{amsthm}
\usepackage{amssymb}
\usepackage[squaren,Gray]{SIunits}

\usepackage{tabularx,booktabs}
\usepackage{geometry}
\geometry{hmargin=2.5cm,vmargin=3.5cm}

\newcounter{numques}
\setcounter{numques}{0}
\newcommand{\question}[1]{
    \addtocounter{numques}{1}
    \noindent {\large \textbf{Q\thenumques.}\,#1}
}
\newcounter{numexos}                   %Création d'un compteur qui s'appelle numexo
\setcounter{numexos}{0}                %initialisation du compteur
\newcommand{\exercice}[1]{             %Création d'une macro ayant un paramètre
    \setcounter{numques}{0}
    \addtocounter{numexos}{1}              %chaque fois que cette macro est appelée, elle ajoute 1 au compteur numexos
    \noindent   {\Large \textbf{Exercice\,\thenumexos\,:}\,#1} %Met en rouge Exercice et la valeur du compteur appelée par \thenumeexos
    }


    \usepackage[a4paper]{hyperref}
    \usepackage{url}
    \usepackage{listings}
    \usepackage{mymath}
    \usepackage{kinematik}
    \usepackage{defined_vectors}
    \usepackage{graphicx,wrapfig}



    \usepackage{pgf, tikz}
    \usetikzlibrary{quotes,angles}
    \usetikzlibrary{arrows}
    \usetikzlibrary{calc}
    \tikzstyle{arrow}=[draw, -latex]


    %%%%%%%%%%%%%%%%%%%%%%%%%%%%%%%%%%%%%%%%%%%%%%%%%%%%%%%%%%%%%%%%%%%%%%%%%%%%%%%%%%%%%%%%%%%%%%%%%%%%%%%%%%%%%%%
    %%%%%%%%%%%%%%%%%%%%%%%%%%%%%%%%%%%%%%%%%%%%%%%%%%%%%%%%%%%%%%%%%%%%%%%%%%%%%%%%%%%%%%%%%%%%%%%%%%%%%%%%%%%%%%%


    \title{\textsc{INGE1-{S1} -- Initiation au Réseau de Neurones }\\
    \emph{TD 1 -- Apprentissage Profond (XOR) } }
    \author{ESME Sudria -- Décembre 2017}
    \date{}%L3 Info -- printemps 2012--2013}




\begin{document}
\maketitle
\pagestyle{empty}

\def\layersep{2.5cm}

\begin{tikzpicture}[shorten >=1pt,->,draw=black!50, node distance=\layersep]

    \tikzstyle{every pin edge}=[<-,shorten <=1pt]
    \tikzstyle{neuron}=[circle,fill=black!25,minimum size=17pt,inner sep=0pt]
    \tikzstyle{input neuron}=[neuron, fill=green!50];
    \tikzstyle{output neuron}=[neuron, fill=red!50];
    \tikzstyle{hidden neuron}=[neuron, fill=blue!50];
    \tikzstyle{annot} = [text width=4em, text centered]

    % Draw the input layer nodes
    \foreach \name / \y in {1,...,4}
    % This is the same as writing \foreach \name / \y in {1/1,2/2,3/3,4/4}
        \node[input neuron, pin=left:Input \#\y] (I-\name) at (0,-\y) {};

    % Draw the hidden layer nodes
    \foreach \name / \y in {1,...,5}
        \path[yshift=0.5cm]
            node[hidden neuron] (H-\name) at (\layersep,-\y cm) {};

    % Draw the output layer node
    \node[output neuron,pin={[pin edge={->}]right:Output}, right of=H-3] (O) {};

    % Connect every node in the input layer with every node in the
    % hidden layer.
    \foreach \source in {1,...,4}
        \foreach \dest in {1,...,5}
            \path (I-\source) edge (H-\dest);

    % Connect every node in the hidden layer with the output layer
    \foreach \source in {1,...,5}
        \path (H-\source) edge (O);

    % Annotate the layers
    \node[annot,above of=H-1, node distance=1cm] (hl) {Hidden layer};
    \node[annot,left of=hl] {Entrée};
    \node[annot,right of=hl] {Sortie};
\end{tikzpicture}
% End of code
\end{document}
